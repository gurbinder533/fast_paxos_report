%%%%%%%%%%%%%%%%%%%%%%%%%%%%%%%%%%%%%%%%%
% University Assignment Title Page 
% LaTeX Template
% Version 1.0 (27/12/12)
%
% This template has been downloaded from:
% http://www.LaTeXTemplates.com
%
% Original author:
% WikiBooks (http://en.wikibooks.org/wiki/LaTeX/Title_Creation)
%
% License:
% CC BY-NC-SA 3.0 (http://creativecommons.org/licenses/by-nc-sa/3.0/)
% 
% Instructions for using this template:
% This title page is capable of being compiled as is. This is not useful for 
% including it in another document. To do this, you have two options: 
%
% 1) Copy/paste everything between \begin{document} and \end{document} 
% starting at \begin{titlepage} and paste this into another LaTeX file where you 
% want your title page.
% OR
% 2) Remove everything outside the \begin{titlepage} and \end{titlepage} and 
% move this file to the same directory as the LaTeX file you wish to add it to. 
% Then add \input{./title_page_1.tex} to your LaTeX file where you want your
% title page.
%
%%%%%%%%%%%%%%%%%%%%%%%%%%%%%%%%%%%%%%%%%
%\title{Title page with logo}
%----------------------------------------------------------------------------------------
%	PACKAGES AND OTHER DOCUMENT CONFIGURATIONS
%----------------------------------------------------------------------------------------

\documentclass[12pt]{article}
\usepackage[english]{babel}
\usepackage[utf8x]{inputenc}
\usepackage{amsmath}
\usepackage{graphicx}
\usepackage{url}
\usepackage{textcomp}

\usepackage[colorinlistoftodos]{todonotes}

\begin{document}

\begin{titlepage}

\newcommand{\HRule}{\rule{\linewidth}{0.5mm}} % Defines a new command for the horizontal lines, change thickness here

\center % Center everything on the page
 
%----------------------------------------------------------------------------------------
%	HEADING SECTIONS
%----------------------------------------------------------------------------------------

\textsc{\LARGE University of Texas at Austin}\\[1.5cm] % Name of your university/college
%\textsc{\Large Porting Fast Paxos}\\[0.5cm] % Major heading such as course name
\textsc{\large CS380L Advanced Operating Systems}\\[0.5cm] % Minor heading such as course title

%----------------------------------------------------------------------------------------
%	TITLE SECTION
%----------------------------------------------------------------------------------------

\HRule \\[0.4cm]
{ \huge \bfseries Porting Fast Paxos}\\[0.4cm] % Title of your document
\HRule \\[1.5cm]
 
%----------------------------------------------------------------------------------------
%	AUTHOR SECTION
%----------------------------------------------------------------------------------------

\begin{minipage}{0.4\textwidth}
\begin{flushleft} \large
\emph{Authors:}\\
Ankit \textsc{Goyal} \\
Cheng \textsc{Fu} \\
Gurbinder \textsc{Gill}
\end{flushleft}
\end{minipage}
~
\begin{minipage}{0.4\textwidth}
\begin{flushright} \large
\emph{Supervisor:} \\
Dr. Emmett \textsc{Witchel} % Supervisor's Name
\end{flushright}
\end{minipage}\\[2cm]

% If you don't want a supervisor, uncomment the two lines below and remove the section above
%\Large \emph{Author:}\\
%John \textsc{Smith}\\[3cm] % Your name

%----------------------------------------------------------------------------------------
%	DATE SECTION
%----------------------------------------------------------------------------------------

{\large \today}\\[2cm] % Date, change the \today to a set date if you want to be precise

%----------------------------------------------------------------------------------------
%	LOGO SECTION
%----------------------------------------------------------------------------------------

\includegraphics[width=40mm]{uni_logo.png}\\[1cm] % Include a department/university logo - this will require the graphicx package
 
%----------------------------------------------------------------------------------------

\vfill % Fill the rest of the page with whitespace

\end{titlepage}


\begin{abstract}
Paxos is a protocol for solving consensus in an asynchronous environment that admits crash failures. \textit{Consensus} is a process of agreeing on one result among group of participants. \textit{Classic Paxos}\cite{lamport} proceeds over several rounds to decide on a sequence of commands. In this paper, we port an existing implementation\cite{libfastpaxos} variant of \textit{classic paxos} called \textit{Fast Paxos}\cite{fastpaxos}. \textit{Fast Paxos} reduces the number of messages between client request and response by 2 \footnote{in case of no conflicts}.
\end{abstract}

\section{Introduction}

The problem of agreeing on a sequence of operations or values proposed by different processes is known as the \textit{Distributed Consensus} problem. From \textit{FLP} result\cite{flpresult}, it is impossible to solve consensus in an asynchronous distributed system even if a single process fails by permanently stopping or if a distributed system suffers a \textit{Partial Failure}, in which processes may stop and recover later. Paxos is an algorithm that gets around this problem by making sure that the system doesn't violate the safety requirements during periods when system behaves asynchronously and is certain to make progress (liveliness) if the system behaves partially synchronously for periods long enough to satisfy the progress requirements.\\

\noindent Classic Paxos and Fast Paxos are most widely studied algorithmic solutions to the problem of distributed consensus. Fast Paxos has smaller theoretical latency, therefore is faster, but Classic Paxos is more resilient and hence can tolerate more failures.
In this project we ported the Fast Paxos algorithm on the simulator to study its behavior in the presence of different failure scenarios. 

\section{Protocal Overview}
\label{sec:examples}

\subsection{Paxos Roles}
In our implementation of Fast Paxos, each server can take following roles:

\begin{enumerate}
\item {\textbf{Proposers} receives request from clients, associate it with the current slot number and send accept $\langle req, slot_num, ballot$\rangle.}

\item {\textbf{Acceptors} that accept proposals}
\item {\textbf{Leader} A distinct proposer assumers the role of leader.}
\end{enumerate}

%there are proposers that propose values, acceptors that accept proposals and learners that learn the outcome of the consensus. Servers can take on all three roles. \\

\noindent A distinct proposer assumes the role of leader. In Classic Paxos, the leader is responsible for serializing the commands in global order by assigning an unique timestamp or slot number when proposing a new value. While in Fast Paxos, the leader is responsible for proposing values when a conflict occurs, that is, when two proposers are trying to propose different values to the same command slot. Leader is checking the progress of the protocol periodically and doing arbitration if it sees a conflict. To be able to detect the conflicts, the leader must also be a learner.    

\subsection{Message Flows}
\noindent The normal-case communication pattern in Paxos protocol is proposer → leader → acceptors → learners. In Fast Paxos, the proposer send its proposal to acceptors directly. bypassing the leader and saving one message delay. So the message flow is proposer → acceptors → learners. However. 
\\

\noindent 


Use section and subsection commands to organize your document. \LaTeX{} handles all the formatting and numbering automatically. Use ref and label commands for cross-references.

\subsection{Comments}

Comments can be added to the margins of the document using the \todo{Here's a comment in the margin!} todo command, as shown in the example on the right. You can also add inline comments too:

\todo[inline, color=green!40]{This is an inline comment.}

\subsection{Tables and Figures}

Use the table and tabular commands for basic tables --- see Table~\ref{tab:widgets}, for example. You can upload a figure (JPEG, PNG or PDF) using the files menu. To include it in your document, use the includegraphics command as in the code for Figure~\ref{fig:frog} below.

% Commands to include a figure:
\begin{figure}
\centering
\includegraphics[width=0.5\textwidth]{frog.jpg}
\caption{\label{fig:frog}This is a figure caption.}
\end{figure}

\begin{table}
\centering
\begin{tabular}{l|r}
Item & Quantity \\\hline
Widgets & 42 \\
Gadgets & 13
\end{tabular}
\caption{\label{tab:widgets}An example table.}
\end{table}

\subsection{Mathematics}

\LaTeX{} is great at typesetting mathematics. Let $X_1, X_2, \ldots, X_n$ be a sequence of independent and identically distributed random variables with $\text{E}[X_i] = \mu$ and $\text{Var}[X_i] = \sigma^2 < \infty$, and let
$$S_n = \frac{X_1 + X_2 + \cdots + X_n}{n}
      = \frac{1}{n}\sum_{i}^{n} X_i$$
denote their mean. Then as $n$ approaches infinity, the random variables $\sqrt{n}(S_n - \mu)$ converge in distribution to a normal $\mathcal{N}(0, \sigma^2)$.

\subsection{Lists}

You can make lists with automatic numbering \dots

\begin{enumerate}
\item Like this,
\item and like this.
\end{enumerate}
\dots or bullet points \dots
\begin{itemize}
\item Like this,
\item and like this.
\end{itemize}

We hope you find write\LaTeX\ useful, and please let us know if you have any feedback using the help menu above.


\begin{thebibliography}{100} 
\bibitem{lamport} Lamport, Leslie (May 1998). "The Part-Time Parliament" \emph{ ACM Transactions on Computer Systems 16 (2): 133–169}
\bibitem{libfastpaxos} Marco Primi and Daniele Sciascia. "libfastpaxos" \emph{\url{http://libpaxos.sourceforge.net/paxos_projects.php#libfastpaxos}}
\bibitem{fastpaxos} Lamport, Leslie (July 2005). "Fast Paxos" \emph{}
\bibitem{flpresult} Fischer, Michael J; Nancy A. Lynch; Michael S. Paterson (April 1985)  "Impossibility of distributed consensus with one faulty process". \emph{Journal of the ACM 32}
\end{thebibliography} 

\end{document}